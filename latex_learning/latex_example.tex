\documentclass{article}

%%%%%%%%%%%%%%%% Packages %%%%%%%%%%%%%%%%
\usepackage{style} % style.sty: LaTeX Document style
\usepackage{graphicx} % graphicx: import external graphics
\usepackage[colorlinks,linkcolor=black,anchorcolor=black,citecolor=black]{hyperref} % hyperref: create hyperlinks within the document
\usepackage{listings} % listings: add non-formatted code text
\usepackage{color, xcolor} % xcolor: provide more predefined colors
\usepackage[comma,numbers,square,sort&compress]{natbib} % natbib: author-data citation styles
\usepackage{amsmath} % amsmath: mathtool package
\usepackage{amssymb} % amssymb: provide more mathematical symbols
\usepackage{amsfonts} % amsfonts: provide more mathematical symbols' font styles
\usepackage{mathrsfs} % mathrsfs: other mathematical symbols.
\usepackage{bbm} % bbm: make \mathbb compatible to lowercase letters
\usepackage{amsthm} % amsthm: introduce the theorem environment
\usepackage{appendix} % appendix: add a separate appendix

%%%%%%%%%%%%%%% New Commands %%%%%%%%%%%%%
% add new command when needed
\newcommand{\mX}{\bf{X}} % define the matrix form by Boldface typeface
\newcommand{\vx}{\bf{x}} % define the vector form
\newcommand{\mf}[1]{\boldmath{$#1$}} % math bold font in line
\newcommand{\EqDef}{\ensuremath{\stackrel{\mathrm{def}}{=}}} % definition equation


%%%%%%%%%%%%%%% New Theorem %%%%%%%%%%%%%%
\newtheorem{theorem}{Theorem}
\newtheorem{lemma}{Lemma}
\newtheorem{definition}{Definition}
\newtheorem{remark}{Remark}
\newtheorem{example}{Example}
\newtheorem{algorithm}{Algorithm}
\newtheorem{corollary}{Corollary}
\newtheorem{proposition}{Proposition}

%%%%%%%%%%%%%%% Document info %%%%%%%%%%%%
\title{A Latex Example}
\name{name \thanks{Thanks}}
\address{address}

%%%%%%%%%%%%%%%%%%%% Main %%%%%%%%%%%%%%%%
\begin{document}
    \maketitle % add title

    \begin{abstract}
        This paper gives a example for every used package.
    \end{abstract}

    \begin{keywords}
        Latex, package, example
    \end{keywords}

    \section{Introduction}
    Used packages including amsmath, amssymb, amsfonts, mathrsfs, bbm, amsthm, appendix, listings, graphicxd, hyperref...

    \section{Examples}
    Examples

    \subsection{Theorems}
    We can add lemma, corollary, proposition similarly.
    \begin{theorem}
        This is a theorem!
        \label{theorem_1}
    \end{theorem}
    \begin{proof}
        This is a proof of the theorem.
    \end{proof}

    \subsection{Math packages}
    Math formulation is supported by amsmath, amssymb, amsfonts, mathrsfs, bbm, amsthm.
    This is a math formulation
    \begin{align}
        F(f)(\xi)=\int_{\bf{R}^n} f(x)\rm e^{-\rm i x\xi}\rm dx,\quad \xi\in\bf{R}^n. \tag{1}
    \end{align}

    This is a matrix
    $$
    \begin{pmatrix}
        1 & 2 & 3\\
        2 & 4 & 6
    \end{pmatrix}.
    $$
    
    \subsection{Listings}
    Package Listings adds code block into paper.
    This is a code block with a frame.
    \begin{lstlisting}[frame = single]
        code line 1
        code line 2
    \end{lstlisting}

    \subsection{Hyperref}
    This package can add hyperlinks for any reference in paper or any citation.
    Such as \autoref{theorem_1} we labeled before and \cite{Example}. We can refer any label we defined, equations, reference...

    \subsection{Graphicx}
    Package graphicx can add images as \autoref{fig_1}

    \begin{figure}[ht]
        \centering % certer image
        \includegraphics[width=\linewidth]{example}
        \caption{Example image} % image description
        \label{fig_1}
    \end{figure}
    
    Or you can align images in a row like \autoref{fig_2}.

    \begin{figure}[htb]
        \centering
        \begin{minipage}[t]{0.48\linewidth}
            \centering
            \includegraphics[width=\linewidth]{example}
            (a) Result 1
        \end{minipage}
        \begin{minipage}[t]{0.48\linewidth}
            \centering
            \includegraphics[width=\linewidth]{example}
            (b) Results 2
        \end{minipage}
   
        \caption{Example of images in a row.}
        \label{fig_2}
    \end{figure}

    Also you align images in a matrix like %\autoref{fig_3}

    \begin{figure}[ht]
        \centering
        \begin{minipage}[htb]{0.48\linewidth}
            \centering
            \includegraphics[width=0.98\linewidth]{example}
            (a)
            \includegraphics[width=0.98\linewidth]{example}
            (c)
        \end{minipage}
        \begin{minipage}[htb]{0.48\linewidth}
            \centering
            \includegraphics[width=0.98\linewidth]{example}
            (b)
            \includegraphics[width=0.98\linewidth]{example}
            (d )
        \end{minipage}
        \caption{Example of images in matrix.}
        \label{fig_3}
    \end{figure}


    \section{Conclusion}
    This is a conclusion.
    
    
    \appendixpage % adds a separate title “Appendices” above the first appendix
    \appendix
    This is a appendix

    \bibliographystyle{bst}
    \bibliography{refs}
\end{document}
